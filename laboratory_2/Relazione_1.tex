\documentclass[a4paper]{article}

\title{\textbf{Relazione attività di laboratorio}\\{\normalsize - Esercitazione 2 -}}
\author{Andrea Lops\\
		Paolo Rotolo\\
		Laura Loperfido\\
		Teresa Pantone
	   }
		
\date{19/12/2018}

\usepackage{graphicx}
\usepackage{listings}

\begin{document}

\maketitle

\lstset{language=Matlab}

\section {Incertezza su grandezze dimensionali}
\subsection{Misure con centimetro estensibile}
Sapendo che l'incertezza sulle misure col centimetro è:
\begin{Large} 
	\begin{equation}
		U_{l} = 2mm
	 \end{equation}
\end{Large}
Si è calcolata lincertezza delle relative misure: 
\begin{Large} 
	\begin{equation}
		l_a = 1180\pm2mm
	 \end{equation}
	 \begin{equation}
		l_v = 1000\pm2mm
	 \end{equation}
\end{Large}
\subsection{Misure col calibro digitale}
Per quanto riguarda invece le misure effettuate col calibro digitale si ha: 
\begin{Large} 
	\begin{equation}
		U_{calibro} = 0.01 + |0.02| = 0.03mm
	 \end{equation}
\end{Large}
Si è calcolata lincertezza delle relative misure: 
\begin{Large} 
	\begin{equation}
		w = 30.08\pm0.03mm
	 \end{equation}
	 \begin{equation}
		h = 3.05\pm0.03mm
	 \end{equation}
\end{Large}
\subsection{Misure di superfici}
Successivamente sono state effettuate le misure indirette con i dati acquisiti:
\begin{Large} 
	\begin{equation}
		S = h*w = 91.744 mm^2
	 \end{equation}
\end{Large}
Con relativa incertezza:
\begin{Large} 
	\begin{equation}
		U_{S} =S(\frac{U_h}{h}+\frac{U_w}{w})
	 \end{equation}
\end{Large}
E quindi: 
\begin{Large} 
	\begin{equation}
		S =91.7\pm 1.0mm^2
	 \end{equation}
\end{Large}
Invece:
\begin{Large} 
	\begin{equation}
		S_L= 2l_a(h+w) = 78186.8 mm^2
	 \end{equation}
\end{Large}
Con relativa incertezza:
\begin{Large} 
	\begin{equation}
		U_{S_L} =2S_L\frac{U_{l_a}}{l_a}+2S_L\frac{(U_h+U_w)}{h+w}
	 \end{equation}
\end{Large}
E quindi: 
\begin{Large} 
	\begin{equation}
		S_L =78200.0\pm 500.0mm^2
	 \end{equation}
\end{Large}
\subsection{Stima del valore del provino in rame}
Usando la formula forniteci e i risultati ottenuti precedentemente è stato possibile misurare:
\begin{Large}
	\begin{equation}
		R_{X_{m}} =\frac{\varphi*l_v}{s}= 0.00019183815835 \Omega
	 \end{equation}
\end{Large}
Con la relativa incertezza
\begin{Large}
	
	\begin{equation}
		U_{R_{X_{m}}} = \varphi\frac{U{l_v}}{l_v}+\varphi\frac{U_{S}}{S}
	 \end{equation}
	 \begin{equation}
		R_{X_{m}} = (19183815835 \pm 4)10^{-14} \Omega
	 \end{equation}
\end{Large}


\section{Dimensionamento $I_{max}$}
Con il seguente programma in Matlab è stato possibile definire $I_{max}$

\lstinputlisting{main.m}

Con risultato: 
\begin{Large}
	\begin{equation}
		I_{max} = 20.82 A
	 \end{equation}
\end{Large}


\section {Incertezza su grandezze elettriche, metodo voltamperometrico}
Per effettuate le misure seguenti sono stati adoperati: il multimetro da banco \emph{Agilent 34410} e il multimetro palmare \emph{U/U1253B}.
\subsection{Misura diretta della resistenza mediante due morsetti}
Adoperando due morsetti si è arrivati alla misurazione della seguente resistenza:
\begin{Large}
	\begin{equation}
		R_{X2W}= 0.189 \Omega
	\end{equation}
	\begin{equation}
		U_{R_{X2W}} = [\pm 0.01\%rdg \pm 0.004\% FSO]
	 \end{equation}
	 \begin{equation}
		R_{X2W} = 0.189 \pm 0.004 \Omega
	 \end{equation}
\end{Large}
Purtroppo però la misura è \emph{errata}: a causa delle resistenze dei due puntali, infatti esse come tutta la cavetteria hanno una resistenza interna.
Il problema è fortunatamnete di facile risoluzione, compensando i puntali stessi. Ed ecco che si ottiene una misurazione più precisa.

\begin{Large}
	\begin{equation}
		R_{X2W_{2}} = 0.007 \Omega
	 \end{equation}
	\begin{equation}
		U_{R_{X2W_{2}}} = [\pm 0.01\%rdg \pm 0.004\% FSO]
	 \end{equation}
	 \begin{equation}
		R_{X2W_{2}} = 0.007 \pm 0.004 \Omega
	 \end{equation}
\end{Large}
\subsection{Misura della resistenza mediante quanto morsetti}
Quello che si è andato a misurare con quattro puntali è solamente la \emph{caduta di tensione} nel resistore: infatti il flusso di corrente misurato nei puntali è pressochè nullo:
\begin{Large}
	\begin{equation}
		R_{X4W}= 0.1909 \Omega
	\end{equation}
	\begin{equation}
		U_{R_{X4W}} = [\pm 0.01\%rdg \pm 0.004\% FSO]
	 \end{equation}
	 \begin{equation}
		R_{X4W} = 0.1909\pm 0.006 \Omega
	 \end{equation}
\end{Large}
\subsection{Misura di resistenza con il metodo amperometrico}
Per ottenere la resistenza $R_{X_{VA}}$ si è andati a calcolare prima $I_{VA}$ e poi $V_{VA}$
\begin{Large}
	\begin{equation}
		I_{VA}= 4.9433 A
	\end{equation}
	\begin{equation}
		U_{I_{VA}} = [\pm 0.3\%rdg \pm 10\% FSO]
	 \end{equation}
	 \begin{equation}
		I_{VA} = 4.9 \pm 0.5 A
	 \end{equation}
\end{Large}

\begin{Large}
	\begin{equation}
		V_{VA}= 0.96 mV
	\end{equation}
	\begin{equation}
		U_{V_{VA}} = [\pm 0.005\%rdg \pm 0.0035\% FSO]
	 \end{equation}
	 \begin{equation}
		V_{VA} = 0.96 \pm 0.004 mV
	 \end{equation}
\end{Large}
E successivamente la resistenza:
\begin{Large}
	\begin{equation}
		R_{X_{VA}}= \frac{V_{VA}}{I_{VA}} =0.000194 \Omega
	\end{equation}
	\begin{equation}
		U_{R_{X_{VA}}} = (\frac{U_{V_{VA}}}{V_{VA}}+\frac{U_{I_{VA}}}{I_{VA}})R_{X_{VA}}
	 \end{equation}
	 \begin{equation}
		R_X{_{VA}} = 0.00019\pm 0.00002\Omega
	 \end{equation}
\end{Large}

\section {Incertezza su grandezze elettriche, metodo di confronto delle cadute di tensione}
È stato d'apprima impostato il multimetro come voltmetro e successivamente sono state misurate le tensioni sul provino e sul campione:
Prendiamo l'incertezza che ci riguarda per il multimetro:

\begin{Large} 
	\begin{equation}
		U_{V} = [\pm 0.0050\%rdg \pm 0.0035\% FSO]
	 \end{equation}
\end{Large}
E quindi si ottiene:
\begin{Large} 
	\begin{equation}
		V_X = 0.817 \pm 0.004 mV
	 \end{equation}
	 \begin{equation}
		V_C = 0.503 \pm 0.004 mV
	 \end{equation}
\end{Large}
\subsection{Calcolo del rapporto delle tensioni}
\begin{Large}
	\begin{equation}
		r = {V_X}/{V_C}= 1.624
	 \end{equation}
\end{Large}
\begin{Large} 
	 \begin{equation}
		U_{r} = U_{O}|\frac{1}{V_X}-\frac{1}{V_C}|+(U_{inl}+U_q)(\frac{1}{|V_X|}+\frac{1}{|V_C|})
	 \end{equation}
\end{Large}
Definiamo prima:
\begin{Large}
	\begin{equation}
		U_O= 0
	 \end{equation}
	 \begin{equation}
		U_{inl} = 3.49 \mu V
	 \end{equation}
	 \begin{equation}
		U_{q} = \frac{Q}{2}= \frac{FS}{2(n\ digit)}=10nV
	 \end{equation}
\end{Large}
\begin{Large} 
	\begin{equation}
		r = 1.624 \pm 0.011
	 \end{equation}
\end{Large}
\subsection{Calcolo del valore della resistenza del provino}
\begin{Large} 
	\begin{equation}
		R_{X_{CdT}} = R_c* r = 
	 \end{equation}
\end{Large}
\subsection{Calcolo della resistività del rame}


\end{document}