\documentclass[a4paper]{article}

\title{\textbf{Relazione attività di laboratorio}\\{\normalsize - Esercitazione 2 -}}
\author{Andrea Lops\\
		Paolo Rotolo\\
		Laura Loperfido\\
		Teresa Pantone
	   }
		
\date{19/12/2018}

\usepackage{graphicx}

\begin{document}

\maketitle

\section {Incertezza su grandezze dimensionali}

Sapendo che l'incertezza sulle misure col centimetro è:
\begin{Large} 
	\begin{equation}
		U_{l} = 2mm
	 \end{equation}
\end{Large}
Si è calcolata lincertezza delle relative misure: 
\begin{Large} 
	\begin{equation}
		l_a = 1180\pm2mm
	 \end{equation}
	 \begin{equation}
		l_v = 1000\pm2mm
	 \end{equation}
\end{Large}
Per quanto riguarda invece le misure effettuate col calibro digitale si ha: 
\begin{Large} 
	\begin{equation}
		U_{calibro} = 0.01 + |0.02| = 0.03mm
	 \end{equation}
\end{Large}
Si è calcolata lincertezza delle relative misure: 
\begin{Large} 
	\begin{equation}
		w = 30.08\pm0.03mm
	 \end{equation}
	 \begin{equation}
		h = 3.05\pm0.03mm
	 \end{equation}
\end{Large}
Successivamente sono state effettuate le misure indirette:
\begin{Large} 
	\begin{equation}
		S = h*w = 91.744 mm^2
	 \end{equation}
\end{Large}
Con relativa incertezza:
\begin{Large} 
	\begin{equation}
		U_{S} =S(\frac{U_h}{h}+\frac{U_w}{w})
	 \end{equation}
\end{Large}
E quindi: 
\begin{Large} 
	\begin{equation}
		S =91.7\pm 1.0mm^2
	 \end{equation}
\end{Large}
Invece:
\begin{Large} 
	\begin{equation}
		S_L= 2l_a(h+w) = 78186.8 mm^2
	 \end{equation}
\end{Large}
Con relativa incertezza:
\begin{Large} 
	\begin{equation}
		U_{S_L} =2S_L\frac{U_{l_a}}{l_a}+2S_L\frac{(U_h+U_w)}{h+w}
	 \end{equation}
\end{Large}
E quindi: 
\begin{Large} 
	\begin{equation}
		S_L =78200.0\pm 500.0mm^2
	 \end{equation}
\end{Large}

\begin{Large}
	\begin{equation}
		R_{X_{m}} =\frac{\varphi*l_v}{s}= 0.00019183815835 \Omega
	 \end{equation}
	\begin{equation}
		U_{R_{X_{m}}} = \varphi\frac{U{l_v}}{l_v}+\varphi\frac{U_{S}}{S}
	 \end{equation}
	 \begin{equation}
		R_{X_{m}} = (19183815835 \pm 4)10^{-14} \Omega
	 \end{equation}
\end{Large}

\section {Incertezza su grandezze elettriche, metodo voltamperometrico}

\begin{Large}
	\begin{equation}
		R_{X2W}= 1.89 \Omega
	\end{equation}
	\begin{equation}
		U_{R_{X2W}} = [\pm 0.01\%rdg \pm 0.004\% FSO]
	 \end{equation}
	 \begin{equation}
		R_{X2W} = 1.89 \pm 0.004 \Omega
	 \end{equation}
\end{Large}

\begin{Large}
	\begin{equation}
		R_{X2W_{2}} = 0.007 \Omega
	 \end{equation}
	\begin{equation}
		U_{R_{X2W_{2}}} = [\pm 0.01\%rdg \pm 0.004\% FSO]
	 \end{equation}
	 \begin{equation}
		R_{X2W_{2}} = 0.007 \pm 0.004 \Omega
	 \end{equation}
\end{Large}

\begin{Large}
	\begin{equation}
		R_{X4W}= 0.1909 \Omega
	\end{equation}
	\begin{equation}
		U_{R_{X4W}} = [\pm 0.01\%rdg \pm 0.004\% FSO]
	 \end{equation}
	 \begin{equation}
		R_{X4W} = 0.1909\pm 0.006 \Omega
	 \end{equation}
\end{Large}

\begin{Large}
	\begin{equation}
		I_{VA}= 4.9433 A
	\end{equation}
	\begin{equation}
		U_{I_{VA}} = [\pm 0.3\%rdg \pm 10\% FSO]
	 \end{equation}
	 \begin{equation}
		I_{VA} = 4.9433 \pm 0.5 A
	 \end{equation}
\end{Large}

\begin{Large}
	\begin{equation}
		V_{VA}= 0.96 mV
	\end{equation}
	\begin{equation}
		U_{V_{VA}} = [\pm 0.005\%rdg \pm 0.0035\% FSO]
	 \end{equation}
	 \begin{equation}
		V_{VA} = 0.96 \pm 0.004 mV
	 \end{equation}
\end{Large}

\begin{Large}
	\begin{equation}
		R_{X_{VA}}= 0.000194 \Omega
	\end{equation}
	\begin{equation}
		U_{R_{X_{VA}}} = [\pm 0.010\%rdg \pm 0.004\% FSO]
	 \end{equation}
	 \begin{equation}
		R_X{_{VA}} = 0.000194 \pm 0.004 \Omega
	 \end{equation}
\end{Large}

\section {Incertezza su grandezze elettriche, metodo di confronto delle cadute di tensione}

Prendiamo l'incertezza che ci riguarda per il multimetro Agilent 34410:

\begin{Large} 
	\begin{equation}
		U_{V} = [\pm 0.0050\%rdg \pm 0.0035\% FSO]
	 \end{equation}
\end{Large}
E quindi si ottiene:
\begin{Large} 
	\begin{equation}
		U_{V_X} = 0.817 \pm 0.00354085 mV
	 \end{equation}
	 \begin{equation}
		U_{V_C} = 0.503 \pm 0.00352515 mV
	 \end{equation}
\end{Large}


\end{document}